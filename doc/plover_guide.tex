% plover_guide.tex

\documentclass[11pt]{article}

\usepackage{courier}  % Pretty code listings.
\usepackage{listings} % Pretty code listings.
\usepackage{hyperref} % Hyperlinks for references and table of contents.
\usepackage{url}      % Formatting and linking of URLs.

\pagestyle{plain}

\newcommand{\plover}{\emph{Plover}}
\newcommand{\version}{2.2.0}
\newcommand{\code}[1]{\texttt{#1}}
\newcommand{\terminal}{\href{https://help.ubuntu.com/community/UsingTheTerminal}{terminal window}}
\newcommand{\meta}[1]{\{#1\}}

\begin{document}
\setlength{\parindent}{0pt} % No indentation.

\title{Plover: Open Source Stenography Software}
\author{Joshua Harlan Lifton}
\date{\today}

\maketitle

\section{Quick Start}

\subsection{Installation}
The latest release of \plover~can be downloaded from
\url{https://launchpad.net/plover} as a compressed \code{.tar.gz}
archive that must be extracted. Among the extracted files will be a
README.txt file. Follow the directions in this file for installing and
running \plover.

\subsection{Starting the Program}
Once installed, \plover~can be run from either the application menu or
from a \terminal~(\code{plover} and then \code{ENTER}). If a stenotype
machine other than the default is used, set the machine type in the
configuration dialog.

\subsection{Configuration}
\label{config}

\plover~comes with a default configuration that should work for most
setups that use a keyboard as the stenotype machine. \plover~can be
configured by clicking the \code{Configure...} button. Within the
configuration dialog, the user can select which stenotype machine to
use, which dictionary file to use, whether or not strokes and
translations should be logged to a file, and whether or not
\plover~should immediately start translating when the program is
started. \plover~will need to be restarted before any configuration
changes take effect.

\section{Stenography Dictionary}
At the core of stenography is a dictionary that translates stroke
combinations into words and phrases. The words and phrases can be
English, another language, or a nonsense string of characters. The
words and phrases can also be commands and special keys other than
printable characters.

\subsection{Dictionary Format}
\plover~expects the stenography dictionary file to be in JSON format
and each dictionary entry within the file to be in a specific variant
of RTF/CRE format.

\subsubsection{Single-stroke Entries}
A simple dictionary entry looks like:

\code{"STROKE": "word or phrase",}

For example:
\begin{itemize}
\item \code{"TKO": "do",}
\item \code{"K": "can",}
\item \code{"KWRORPBL": "I don't remember",}
\item \code{"PHAS": "mas",}
\end{itemize}

\subsubsection{Multiple-stroke Entries}
A dictionary entry for more than one stroke looks like:

\code{"STROKE1/STROKE2/STROKE3": "another word or phrase",}

\begin{itemize}
\item \code{"SEUPB/OPB/PHOUS": "synonymous",}
\item \code{"TKE/TPER": "defer",}
\item \code{"TOEPBD/SRA*EU": "tend to have a",}
\item \code{"KWREUP/KWREU/KAOEU/KWRAEU": "yippee ki-yay",}
\item \code{"PEPB/SAR": "pensar",}
\end{itemize}

\subsubsection{Meta Commands}

Meta commands are dictionary entries surrounded by \{ and \}, like:

\code{"STROKE1/STROKE2": "\meta{\^{ }opolis}",}

which means append ``opolis'' to the previous word without inserting a
space.

Meta commands and regular text can be mixed, like:

\code{"STROKE1/STROKE2": "\meta{.}Thus, we can see that",}

which means append a period and the number of spaces after a period
and the text ``Thus, we can see that''.

Spaces between meta commands and regular text don't count, so the
previous example is the same as:

\code{"STROKE1/STROKE2": "\meta{.} Thus, we can see that",}

Furthermore, since the \meta{.} meta command will also capitalize the
next word, the previous example is also the same as:

\code{"STROKE1/STROKE2": "\meta{.} thus, we can see that",}

There can be more than one meta command in a translation, like:

\code{"STROKE1/STROKE2": "\meta{,}like\meta{,}}"

Since \code{"}, \code{\{}, and \code{\}} are all special dictionary
format characters, the character sequences \code{$\backslash$"},
\code{$\backslash$\{}, and \code{$\backslash$\}}, should be used
within a translation, like:

\code{"STROKE1/STROKE2": "\meta{\^{ }.$\backslash$"~}\meta{-\textbar}",}

which will append ." and a space to the previous word and then
capitalize the next word.

\plover~supports the following meta commands:
\begin{itemize}

\item Sentence stops: \meta{.}, \meta{!}, \meta{?} all end a sentence
  with the corresponding punctuation, insert a space and capitalize
  the next letter.

\item Sentence breaks: \meta{,}, \meta{:}, \meta{;} all break a
  sentence with the corresponding punctuation.
  
\item Simple suffixes: \meta{\^{ }ed}, \meta{\^{ }ing}, \meta{\^{ }er},
  \meta{\^{ }s} applies basic orthographic rules to append the
  corresponding prefix to the most recent word to form the past tense,
  present progressive, noun, or plural, respectively.

\item Capitalize: \meta{-\textbar} capitalizes the next letter.

\item Glue flag: \meta{\&X} will connect the string X without a space to any
  adjacent (previous or following) strings that also have glue flags.

\item Attach flag: \meta{\^{ }X}, \meta{X\^{ }}, \meta{\^{ }X\^{ }}
  will connect the string X without a space to the previous, next, or
  both previous and next, respectively, words, unless X is one of the
  simple suffixes (ed, ing, er, s), in which case the simple suffix
  rules take precedence.

\item \plover~controls: \meta{PLOVER:X} will control the state of the
  \plover~program itself, where X is one of \code{SUSPEND},
  \code{RESUME}, \code{TOGGLE}, \code{CONFIGURE}, \code{FOCUS}, or
  \code{QUIT}.

\item Key combinations: \meta{\#X} will execute the key combination
  described by X. See below for details.

\end{itemize}

\subsubsection{Arbitrary Key Combinations}

Meta commands of the form \meta{\#X} are interpreted as a sequence of
keyboard keys pressed and released in sequence and/or
simultaneously. Serial key presses are separated by a single
space. Simultaneous key presses are denoted by parentheses surrounding
keys pressed while another key is held down. For example:

\code{Alt\_L(Tab)}

will emulate the action of holding down the left Alt key while
pressing and releasing the Tab key and then releasing the left Alt
key. Parentheses can be nested, as in:

\code{Control\_L(Shift\_L(minus minus minus)}

which would emulate holding down the left Control key, then holding
down the left Shift key, and then pressing the minus (-) key three
times before releasing the left Shift and then the left Control
keys. 

Below is the list of all legal keys that can be used to form key
combination commands.

\begin{itemize}

\item 0 1 2 3 4 5 6 7 8 9

\item a b c d e f g h i j k l m n o p q r s t u v w x y z

\item A B C D E F G H I J K L M N O P Q R S T U V W X Y Z

\item Alt\_L Alt\_R Control\_L Control\_R Hyper\_L Hyper\_R Meta\_L
  Meta\_R Shift\_L Shift\_R Super\_L Super\_R

\item Caps\_Lock Num\_Lock Scroll\_Lock Shift\_Lock

\item Return Tab BackSpace Delete Escape Break Insert Pause Print Sys\_Req

\item Up Down Left Right Page\_Up Page\_Down Home End

\item F1 F2 F3 F4 F5 F6 F7 F8 F9 F10 F11 F12 F13 F14 F15 F16 F17 F18
  F19 F20 F21 F22 F23 F24 F25 F26 F27 F28 F29 F30 F31 F32 F33 F34 F35

\item L1 L2 L3 L4 L5 L6 L7 L8 L9 L10

\item R1 R2 R3 R4 R5 R6 R7 R8 R9 R10 R11 R12 R13 R14 R15

\item KP\_0 KP\_1 KP\_2 KP\_3 KP\_4 KP\_5 KP\_6 KP\_7 KP\_8 KP\_9
  KP\_Add KP\_Begin KP\_Decimal KP\_Delete KP\_Divide KP\_Down KP\_End
  KP\_Enter KP\_Equal KP\_F1 KP\_F2 KP\_F3 KP\_F4 KP\_Home KP\_Insert
  KP\_Left KP\_Multiply KP\_Next KP\_Page\_Down KP\_Page\_Up KP\_Prior
  KP\_Right KP\_Separator KP\_Space KP\_Subtract KP\_Tab KP\_Up

\item ampersand apostrophe asciitilde asterisk at backslash braceleft
  braceright bracketleft bracketright colon comma division dollar
  equal exclam greater hyphen less minus multiply numbersign parenleft
  parenright percent period plus question quotedbl quoteleft
  quoteright semicolon slash space underscore

\item AE Aacute Acircumflex Adiaeresis Agrave Aring Atilde Ccedilla
  Eacute Ecircumflex Ediaeresis Egrave Eth ETH Iacute Icircumflex
  Idiaeresis Igrave Ntilde Oacute Ocircumflex Odiaeresis Ograve
  Ooblique Otilde THORN Thorn Uacute Ucircumflex Udiaeresis Ugrave
  Yacute

\item ae aacute acircumflex acute adiaeresis agrave aring atilde
  ccedilla eacute ecircumflex ediaeresis egrave eth iacute icircumflex
  idiaeresis igrave ntilde oacute ocircumflex odiaeresis ograve oslash
  otilde thorn uacute ucircumflex udiaeresis ugrave yacute ydiaeresis

\item cedilla diaeresis grave asciicircum bar brokenbar cent copyright
  currency degree exclamdown guillemotleft guillemotright macron
  masculine mu nobreakspace notsign onehalf onequarter onesuperior
  ordfeminine paragraph periodcentered plusminus questiondown
  registered script\_switch section ssharp sterling threequarters
  threesuperior twosuperior yen

\item Begin Cancel Clear Execute Find Help Linefeed Menu Mode\_switch
  Multi\_key MultipleCandidate Next PreviousCandidate Prior Redo
  Select SingleCandidate Undo
  
\item Eisu\_Shift Eisu\_toggle Hankaku Henkan Henkan\_Mode Hiragana
  Hiragana\_Katakana Kana\_Lock Kana\_Shift Kanji Katakana Mae\_Koho
  Massyo Muhenkan Romaji Touroku Zen\_Koho Zenkaku Zenkaku\_Hankaku

\end{itemize}

\subsection{Editing the Dictionary}

To edit the dictionary file, issue the following command in a
\terminal:

\begin{lstlisting}
gedit $HOME/.config/dict.json
\end{lstlisting}

or a comparable command with the location of whatever dictionary file
\plover~ is configured to use. In case of a character encoding error
in \code{gedit}, do the following:
\begin{enumerate}
\item Click the character encoding drop-down list.
\item Select Add or Remove...
\item Select Western ISO-8859-1, also known as latin-1.
\item Click the Add button.
\item Click the OK button.
\item Select ISO-8859-1 from the character encoding drop-down list.
\item Click the Retry button.
\end{enumerate}

\section{Further Resources}
Other than this document, there are several resources for getting help
with \plover:

\begin{itemize}
\item mailing list: \url{http://groups.google.com/group/ploversteno}
\item blog: \url{http://plover.stenoknight.com}
\item IRC channel: \code{\#plover} on \code{irc.freenode.net} or \\
  \url{http://webchat.freenode.net/?channels=#plover}
\item download and development: \url{http://launchpad.net/plover}
\item general information: \url{http://stenoknight.com/plover}
\end{itemize}

\end{document}
